\chapter[Introducción]{Introducción}\label{ch:capitulo1}
%\fpar


% ANTECEDENTES Y MOTIVACION

\section{Motivación}\label{chsub:Motivación}

Nuestra principal motivación es poder generar un descriptor eficiente, en tiempo de ejecución y que sea ligero con el fin de poder ser utilizado, por ejemplo, en aplicaciones móviles. En detalle, dicho descriptor podrá ser utilizado para crear aplicaciones que nos permitan, por ejemplo, pasear por un museo y dentro de éste, visualizar un elemento; y mientras lo observamos obtenemos información sobre este mismo de forma rápida y precisa.


Incluso poder ir por la calle, ver a personas  y poder saber quienes son antes de que nos saluden, detectar objetos interesantes y obtener información sobre ellos, llevar la computación más allá del viejo y olvidado ordenador.

\section{Contexto}\label{chsub:Contexto}

Hoy en día es habitual escuchar a las personas comentar sobre las cosas que hace Facebook o Google en el campo de visión por computador, por ejemplo, Facebook reconoce los rostros de las personas y te recomienda etiquetarlos o te dice ``esta persona puede ser conocido tuyo'' por medio de una fotografía, ó como ``Google Image'' que con tan solo escribir una palabra este te entrega un conjunto de imágenes extraídas de la web relacionadas con la palabra escrita.

Dados estos ejemplos y junto con la posibilidad de investigar nuevas tecnologías relacionadas con esta área de la computación, y tomando en cuenta el estado del arte en el que está este tipo de investigaciones, presentaremos una manera eficiente, ligera y rápida para la identificación de objetos y expresiones faciales.

Nos centraremos en la investigación de un nuevo mecanismo de descripción de características con el fin de crear una estructura que represente eficazmente dichas características y sea ligero para su procesamiento en el reconocimiento de objetos y expresiones.
%
%\section{Título del Subcapítulo 2} \label{chsub:Título del Subcapítulo 2}
%
%\parindent=0pt Lorem ipsum ad his scripta blandit partiendo, eum fastidii accumsan euripidis in, eum liber hendrerit an.
%
%\vspace{0.5cm}
%\parindent=30pt Quo mundi lobortis reformidans eu, legimus senserit definiebas an eos. Eu sit tincidunt incorrupte definitionem, vis mutat affert percipit cu, eirmod consectetuer signiferumque eu per.
%
%\subsection{Título de la sección 1 del subcapítulo 2}\label{chsub:Título de la sección 1 del subcapítulo 2}
%
%\parindent=0pt Eos vocibus deserunt quaestio ei. Blandit incorrupte quaerendum in quo, nibh impedit id vis, vel no nullam semper audiam.
%
